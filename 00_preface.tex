\chapter*{Preface}
I have been searching for a real open source alternative to expensive and proprietary routers and firewalls for a long time. There were basically three types of network solutions:
\begin{itemize}
 \item Proprietary integrated solutions.
 \item General purpose systems with networking software available, where you have to maintain all of packages and
 uncoordinated configuration files manually.
 \item Open source systems focused on one application (usually security or SOHO usage).
\end{itemize}
When I have found Vyatta (version 4.0 by that time), I was very inspired by it, since it was exactly what I have been
looking for. It was open source, it had unified management interface and single configuration file, and it supported 
most part of networking functions. Another good thing was that it was actually the only network operating system able to run in virtualized environment, and virtualization was officially supported.

But, whereas Vyatta is great, there is not so many information sources about it (which is not so surprising, since it is a relatively young project). Official documentation is more like a command reference with limited amount of practical examples, various howto guides and blog articles are useful, but sparse and uncoordinated.

So I decided to write a book with practical configuration examples to summarize currently available information. This book does not assume its reader already has experience working with Vyatta, however it does not explain the bases of networking and thus assumes they have some experience with other network solutions.

Structure of this book is inspired by Aviva Garett's ``JUNOS Cookbook'', an excellent guide to Juniper Networks routers.

\section*{Typography Considerations}
{\tt{ \bf Bold text}} means commands to be typed in console. \\
\texttt{Monospace text} means commands output.
