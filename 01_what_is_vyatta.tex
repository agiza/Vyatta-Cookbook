\chapter{What is Vyatta}
\section{Description and Functions}
Vyatta is an open source routing and security platform. It is based on Debian GNU/Linux distribution. 
Unlike general purpose distributions, it has unified management console (similar to Juniper Networks JUNOS) 
and single configuration file, which makes it easy to configure and maintain. \\
Vyatta runs on x86 hardware and supports various network interface cards. 

Many Vyatta functions are provided by existing open source software, modified or not (e.g. Quagga as a 
routing stack, StrongS/WAN as an IPsec implementation etc.).

By the time of release 7.0 its function list includes the following main items:
\begin{itemize}
 \item IPv4 routing: static, RIPv2, OSPF, BGP;
 \item IPv6 routing: static, RIPng, OSPFv3, BGP, router advertisment;
 \item Network security: firewall for IPv4 and IPv6, intrusion detection system, p2p traffic control, rate limiting,
 time based rules, stateful failover;
 \item IP services: QoS, DHCP server and relay (for IPv4 and IPv6), NAT, web proxy with URL filtering, traffic
 accounting (sFlow and NetFlow);
 \item VPN: IPsec, L2TP/IPsec and PPTP servers, OpenVPN;
 \item Datalink layer: Ethernet, PPPoE, 802.1q VLANs, 802.3ad link aggregation, 802.11[abg] wireless, wireless
 modems;\footnote{Subscription edition also supports serial WAN interfaces such as v.35, T1/T3 and E1.}
 \item Remote access: SSH, telnet, web interface, remote access API\footnote{Remote access API is a subscription
 only feature.};
\end{itemize}

\section{Vyatta Editions}
There are three Vyatta editions with different policy and supported functions:
\begin{itemize}
 \item Vyatta Core (formerly Vyatta Community Edition). Fully open source, no technical support provided, free of
 charge.
 \item Vyatta Subscription. Contains additional components (including closed source), technical support is provided.
 Accessible only for subscription customers.
 \item Vyatta Plus. Similar to Subscription, but includes more value added components. 
\end{itemize}

Both Vyatta Subscription and Vyatta Plus are based on Vyatta Core source code with only difference in additional
components, hence the name of Vyatta Core. Until version 6 they were separate branches, but have been merged, 
that is why the name Community was changed to Core.

\section{Brief Vyatta History}
Vyatta development has started in 2005 by company called Vyatta inc. founded by former employees of major network
solution vendors like Cisco Systems. Currently Vyatta is deployed literally worldwide and used by many enterprises 
from small business to Fortune~500 companies. 