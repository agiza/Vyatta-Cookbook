\chapter{Vyatta Installation}
There are basically three ways of getting Vyatta installed:
\begin{enumerate}
 \item Installation on physical hardware
 \item Installation on a virtual machine
 \item Ordering hardware with preinstalled Vyatta
\end{enumerate}

Vyatta inc. offers some hardware appliances with different form factor and performance and Vyatta Subscription
preinstalled. However, choosing your own hardware for specific purpose or reusing existing hardware may be a better
option. 

Possibility to install on a virtual machine is actually the ``killing feature'' -- for now Vyatta is the only
routing and security software suitable for virtual environments. Citrix XenServer\texttrademark\ and 
VMWare\texttrademark\ hypervisors are officially supported (and images for them are provided), KVM and 
VirtualBox\texttrademark\ are reported to work by community members.

Vyatta is shipped in several forms:
\begin{enumerate}
 \item Generic installation LiveCD
 \item Installation CD for virtual machine
 \item XenServer template
 \item VMWare template
\end{enumerate}

Also there are two installation options: file based and image based. File based installation is similar to how 
general purpose systems are installed. In image based setup all system files are kept in a 
SquashFS\footnote{A read only compressing file system, widely used for LiveCD and similar images.} compressed image
which is united with read/write file system at boot time.

Image based installation is the recommended way. It allows to keep several images on a single machine and thereby 
perform safe upgrade with possibility to rollback to previous version if something went wrong by mere reboot. An image
can be installed on a file based setup too.

\section{Install from .iso image}
\problem{You want to install Vyatta from a LiveCD image.}
\solution
Installation procedure is common for generic and virtualization ISO images. First you need to boot your hardware or
virtual machine from it. Both of them work as LiveCD, so you can try it out without installation (configuration file
can be saved on a floppy drive in that case). To install you will need access to serial or keyboard/video console.

After you booted your machine, you will see the login prompt:
\begin{verbatim}
Welcome to Vyatta - vyatta tty1
vyatta login: 
\end{verbatim}

Log in with user name ``vyatta'' and password ``vyatta''. You will see the command prompt:
\begin{verbatim}
vyatta@vyatta:~$
\end{verbatim}

To perform image based installation, type command \command{install-image} and press Enter key. The installer will ask
you several simple questions. Note that you need at least 2~GB big partition to install.
Here is a sample installation dialog (in triangle brackets user actions are shown):
\begin{verbatim}
vyatta@vyatta:~$ install-image 
Welcome to the Vyatta install program.  This script
will walk you through the process of installing the
Vyatta image to a local hard drive.
Would you like to continue? (Yes/No) [Yes]: <Enter pressed>
Probing drives: OK
Looking for pre-existing RAID groups...none found.
The Vyatta image will require a minimum 1000MB root.
Would you like me to try to partition a drive automatically
or would you rather partition it manually with parted?  If
you have already setup your partitions, you may skip this step

Partition (Auto/Parted/Skip) [Auto]: <Enter pressed>

I found the following drives on your system:
 vda	2147MB


Install the image on? [vda]: <Enter pressed>

This will destroy all data on /dev/vda.
Continue? (Yes/No) [No]: Yes

How big of a root partition should I create? (1000MB - 2147MB) [2147]MB: 

Creating filesystem on /dev/vda1: OK
Done!
Mounting /dev/vda1...
What would you like to name this image? [VC6.1-2010.08.20]: 
OK.  This image will be named: VC6.1-2010.08.20
Copying squashfs image...
Copying kernel and initrd images...
Done!
I found the following configuration files
/opt/vyatta/etc/config/config.boot
Which one should I copy to vda? [/opt/vyatta/etc/config/config.boot]: 

Enter password for administrator account
Enter vyatta password: <password typed, typed keys not visible>
Retype vyatta password: <password retyped, typed keys not visible>
I need to install the GRUB boot loader.
I found the following drives on your system:
 vda	2147MB


Which drive should GRUB modify the boot partition on? [vda]: <Enter pressed>

Setting up grub: OK
Done!
\end{verbatim}
After installation finished, reboot your router.
%\pagebreak

\section{Upgrade an existing system}
\problem{You want to upgrade an existing system to newer software version.}
\solution
With image based installation you may have as many images as you want (and your disk space allows). To add a new
image, enter command ``\command{add system image <URL>}''. An example of image installation dialog:
\begin{verbatim}
vyatta@vyatta:~$ add system image ftp://10.9.7.5/vyatta-livecd_VC6.1-2010.10.16_i386.iso
Welcome to the Vyatta install program.  This script
will walk you through the process of installing the
Vyatta image to a local hard drive.
Would you like to continue? (Yes/No) [Yes]: 
Trying to fetch ISO file from ftp://10.9.7.5/vyatta-livecd_VC6.1-2010.10.16_i386.iso
  % Total    % Received % Xferd  Average Speed   Time    Time     Time  Current
                                 Dload  Upload   Total   Spent    Left  Speed
100  160M  100  160M    0     0  26.9M      0  0:00:05  0:00:05 --:--:-- 22.1M
ISO download suceeded.
Checking for digital signature file...
  % Total    % Received % Xferd  Average Speed   Time    Time     Time  Current
                                 Dload  Upload   Total   Spent    Left  Speed
  0     0    0     0    0     0      0      0 --:--:-- --:--:-- --:--:--     0
curl: (19) RETR response: 550
Unable to fetch digital signature file.
Do you want to continue without signature check? (yes/no) [yes] 
Checking MD5 checksums of files on the ISO image...OK.
You are running an installed system. Do you want to use the current install
partition? (Yes/No) [Yes]: 

Done!
What would you like to name this image? [VC6.1-2010.10.16]: 
OK.  This image will be named: VC6.1-2010.10.16
Installing "VC6.1-2010.10.16" image.
Copying new release files...
Would you like to save the current configuration 
directory and use the current start-up configuration 
for the new version? (Yes/No) [Yes]: 
Copying current configuration...
Would you like to save the SSH host keys from your 
current configuration? (Yes/No) [Yes]: 
Copying SSH keys...
Setting up grub configuration...
Done.
\end{verbatim}
New image is set to default boot automatically. After installation finished, you may see the list of images
by using command ``\command{show system image}'':
\begin{verbatim}
vyatta@vyatta:~$ show system image 
The system currently has the following image(s) installed:

   1: VC6.1-2010.10.16 (default boot)
   2: VC6.1-2010.08.20
\end{verbatim}
Reboot your router to make it use newly installed image.


\section{Delete an unused image}
\problem{You want to delete a system image (for example, a previous version you do not want 
to use after upgrade anymore).}
\solution 
Enter command ``\command{delete system image <name>}''. Use name as it shown in ``\command{show system image}''.
output. Note that you can not delete currently running image. \ 
Here is an example of image deletion dialog:
\begin{verbatim}
vyatta@vyatta:~$ delete system image VC6.1-2010.08.20 
Are you sure you want to delete the
"VC6.1-2010.08.20" image? (Yes/No) [No]: Yes
Deleting the "VC6.1-2010.08.20" image...
Done
\end{verbatim}

\section{Change boot image}
\problem{You want to change default boot image (e.g. to compare newer and older versions).}
\solution
To change default boot image, use command ``\command{set system image default-boot <name>}''.
