\chapter{Vyatta Installation}
There are basically three ways of getting Vyatta installed:
\begin{enumerate}
 \item Installation on physical hardware
 \item Installation on a virtual machine
 \item Ordering hardware with preinstalled Vyatta
\end{enumerate}

Vyatta inc. offers some hardware appliances with different form factor and performance and Vyatta Subscription
preinstalled. However, choosing your own hardware for specific purpose or reusing existing hardware may be a better
option. 

Possibility to install on a virtual machine is actually the ``killing feature'' -- for now Vyatta is the only
routing and security software suitable for virtual environments. Citrix XenServer\texttrademark\ and 
VMWare\texttrademark\ hypervisors are officially supported (and images for them are provided), KVM and 
VirtualBox\texttrademark\ are reported to work by community members.

Vyatta is shipped in several forms:
\begin{enumerate}
 \item Generic installation LiveCD
 \item Installation CD for virtual machine
 \item XenServer template
 \item VMWare template
\end{enumerate}

Also there are two installation options: file based and image based. File based installation is similar to how 
general purpose systems are installed. In image based setup all system files are kept in a 
SquashFS\footnote{A read only compressing file system, widely used for LiveCD and similar images.} compressed image
which is united with read/write file system at boot time.

Image based installation is the recommended way. It allows to keep several images on a single machine and thereby 
perform safe upgrade with possibility to rollback to previous version if something went wrong by mere reboot. An image
can be installed on a file based setup too.

\section{Install from .iso image}
Installation procedure is common for generic and virtualization ISO images. First you need to boot your hardware or
virtual machine from it. Both of them work as LiveCD, so you can try it out without installation (configuration file
can be saved on a floppy drive in that case). To install you will need access to serial or keyboard/video console.

After you booted your machine, you will see the login prompt:
\begin{verbatim}
Welcome to Vyatta - vyatta tty1
vyatta login: 
\end{verbatim}

Log in with user name ``vyatta'' and password ``vyatta''. You will see the command prompt:
\begin{verbatim}
vyatta@vyatta:~$
\end{verbatim}

To perform image based installation, type command \texttt{install-image} and press Enter key. The installer will ask
you several simple questions. Here is a sample installation dialog:
\begin{verbatim}
vyatta@vyatta:~$ install-image 
Welcome to the Vyatta install program.  This script
will walk you through the process of installing the
Vyatta image to a local hard drive.
Would you like to continue? (Yes/No) [Yes]: 
Probing drives: OK
Looking for pre-existing RAID groups...none found.
The Vyatta image will require a minimum 1000MB root.
Would you like me to try to partition a drive automatically
or would you rather partition it manually with parted?  If
you have already setup your partitions, you may skip this step

Partition (Auto/Parted/Skip) [Auto]: 

I found the following drives on your system:
 vda	2147MB


Install the image on? [vda]:

This will destroy all data on /dev/vda.
Continue? (Yes/No) [No]: Yes
This will destroy all data on /dev/vda.
Continue? (Yes/No) [No]: Yes

How big of a root partition should I create? (1000MB - 2147MB) [2147]MB: 

Creating filesystem on /dev/vda1: OK
Done!
Mounting /dev/vda1...
What would you like to name this image? [999.mendocino.11181940]: 
OK.  This image will be named: 999.mendocino.11181940
Copying squashfs image...
Copying kernel and initrd images...
Done!
I found the following configuration files
/opt/vyatta/etc/config/config.boot
Which one should I copy to vda? [/opt/vyatta/etc/config/config.boot]: 

Enter password for administrator account
Enter vyatta password:
Retype vyatta password:
I need to install the GRUB boot loader.
I found the following drives on your system:
 vda	2147MB


Which drive should GRUB modify the boot partition on? [vda]:

Setting up grub: OK
Done!
vyatta@vyatta:~$ 

\end{verbatim}


