\chapter{System Configuration}
Vyatta has various system options controlling user rights, authentication properties, date and time and so on.

\section{Recipe: Configure Host Name}
Probably the first thing you want to do with a new router is to configure its host name to distinguish it from
other hosts in your network. It can be done with command ``\command{set system hostname }''.
\begin{verbatim}
vyatta@vyatta# set system host-name router 
[edit]
vyatta@vyatta# commit
\end{verbatim}
Note that you should re-login to see your changes applied. Newly connected users will see these changes right after
you issue this command and commit changes.

Host name should start with a letter and contain only letters, digits and ``-'' sign.

\section{Recipe: Configure Time Zone}
To see log messages with proper time stamps you may want to configure the time zone. Use 
``\command{set system time-zone <zone>/<city>}'' for it. For instance, ``US/Pacific''.

Use command completion to find your time zone.

Summer time is automatically set for time zones that use it, there is no need to configure it
manually.

\section{Recipe: Configure NTP Server}
By default some NTP servers are already configured to synchronize time with. You may add your own by using command
``\command{set system ntp server <address>}''.

Note that Vyatta also is implicitly functioning as NTP server, so you may configure your clients
to synchronize time with it. If it is not desired behaviour, NTP requests can be blocked with
firewall rules.

\section{Recipe: Configure DNS Server}
To be able to resolve domain names on your router you need to configure a DNS server to use. It can be done
with command ``\command{set system name-server <IP address>}''.

\section{Recipe: Disable IPv4 or IPv6 routing}
You may want to administratively disable routing of IPv4 or IPv6 packets on your system. It can be done
with commands ``\command{set system ip disable-forwarding}'' and ``\command{set system ipv6 disable-forwarding}''.

By default both are enabled.

\section{Recipe: Add New User}
\problem{You want to create several users in addition to default ``vyatta'' user.}
\solution
Use command ``\command{set system login user <name>}''. Then you should set password with command 
``\command{set system login user <name> authentication plaintext-password <password>}'' and may set
full name with ``\command{set system login user <name> full-name "<Name Surname>"}''.

Do not be confused by option name ``plaintext-password''. It just indicates you enter unencrypted password
that will be encrypted on commit. 

If you migrate users from another already configured system and have password hashes, you may
enter them by using option ``encrypted-password'' instead of ``plaintext-password''.

\section{Recipe: Restrict Some User to Operational Commands Only}
If you want to give some to user only privelege of using operational mode, but not of modifying configuration,
you may do it by setting option ``\command{set system login user <name> level operator}''.

If you want to raise privileges of such user in the future, set that option to ``admin'' value.